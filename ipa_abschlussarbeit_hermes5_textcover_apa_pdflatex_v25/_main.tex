% Diese Konfiguration ist nur mit pdfLaTeX-Compiler kompatibel.
\input{config}

% ---------------------------------------------------
%     DOKUMENTMETADATEN (anpassbar durch Nutzer:in)
% ---------------------------------------------------

\newcommand{\titletext}{Anleitung zur Nutzung der LaTeX-Vorlage}
\newcommand{\subtitletext}{Erstellung von Facharbeiten gemäss Agogis-Richtlinien}
\newcommand{\authorname}{Turukmoorea – \protect\href{https://github.com/Turukmoorea}{github.com/Turukmoorea}}
\newcommand{\classinfo}{HF SP Anleitung}
\newcommand{\instituteinfo}{Agogis – nicht offiziell}
\newcommand{\studyprogram}{Version 2025}

% -------------------------------
% Titelblatt und Inhaltsverzeichnis
% -------------------------------

\begin{document}
\begin{titlepage}
  % Hintergrundbild (zentriert, skaliert)
  \frontpageimg[0.75]{300,427}{titelbild_michele_epg.png}  % [scale]{links,oben}{bild}

  % Textblock
  \begingroup
    \vspace*{18cm}
    \begin{flushleft}
      \textcolor{fronttitlecolor}{%
        \fontsize{\fronttitlefontsize}{1.2\fronttitlefontsize}\selectfont \textbf{\titletext}}\\[0.5cm]

      \textcolor{frontsubtitlecolor}{%
        \fontsize{\frontsubtitlefontsize}{1.2\frontsubtitlefontsize}\selectfont \MakeUppercase{\subtitletext}}\\[0.2cm]

      \textcolor{frontauthorcolor}{%
        \fontsize{\frontauthorfontsize}{1.2\frontauthorfontsize}\selectfont \MakeUppercase{\authorname}}%
    \end{flushleft}

    \vspace{2.5cm}

    \noindent
    \makebox[\textwidth]{%
      \begin{minipage}[t]{0.5\textwidth}
        \raggedright
        \textcolor{frontmetacolor}{%
          \fontsize{\frontmetafontsize}{1.2\frontmetafontsize}\selectfont \instituteinfo, \classinfo}%
      \end{minipage}%
      \begin{minipage}[t]{0.5\textwidth}
        \raggedleft
        \textcolor{frontmetacolor}{%
          \fontsize{\frontmetafontsize}{1.2\frontmetafontsize}\selectfont \studyprogram}%
      \end{minipage}%
    }%
  \endgroup
\end{titlepage}

\cleardoublepage
\tableofcontents

\cleardoublepage
% -------------------------------
%     Inhalt des Dokumentes
% -------------------------------

\input{einfuhrung}

\include{grundstruktur}

\input{crashkurs}

\section{APA-Zitation und Quellenverzeichnis}
\subsection{Einrichtung mit \texttt{biblatex} und \texttt{biber}}
\subsection{Quellen einbinden mit \texttt{.bib}-Datei}
\subsection{Zitieren im Text mit \texttt{\textbackslash parencite}}
\subsection{Automatisches Literaturverzeichnis einfügen}

\section{Overleaf - Wie nutze ich das Tool?}

\section{KI Modelle und ihre Wasserzeichen}

\section{Export nach Word (MS Word Abgabe) – der Workaround}
\subsection{Export aus Overleaf als PDF}
\subsection{PDF in Word öffnen (z.B. in Word 2021 oder 365)}
\subsection{Speichern als \texttt{.docx} für Abgabe}
\subsection*{Hinweis}
Dies ist ein rein technischer Workaround für Schulen, die Word verlangen. Inhaltlich sollte mit PDF gearbeitet werden.



% -------------------------------
%             Anhang
% -------------------------------
\cleardoublepage
\section{Anhang}

\begin{itemize}
  \item 
\end{itemize}

% -------------------------------
%      Quellenverzeichnis
% -------------------------------
\cleardoublepage
\section{Quellenverzeichnis}

\printbibliography[heading=none]

\vspace{1cm}

\subsection{Einsatz von KI-basierten Tools}
\begin{itemize}
  \item ChatGPT (OpenAI, Version 4, Zugriff: 03.05.2025): zur Ideenstrukturierung und sprachlichen Überarbeitung.
  \item DeepL (Zugriff: 03.05.2025): zur Übersetzung von Begriffen.
  \item Quillbot: nicht verwendet.
\end{itemize}

% -------------------------------
%      Abbildungsverzeichnis
% -------------------------------
\cleardoublepage
\section{Abbildungsverzeichnis}
\listoffigures

% -------------------------------
%      Tabellenverzeichnis
% -------------------------------
\cleardoublepage
\section{Tabellenverzeichnis}
\listoftables

\end{document}
